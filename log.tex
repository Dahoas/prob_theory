\documentclass[10pt]{article}


\usepackage{amssymb,amsthm,amsmath}
\usepackage{enumerate}
\usepackage{graphicx,color}
\usepackage[hidelinks]{hyperref}
%\usepackage{refcheck}

\newcommand{\dd}{\mathrm{d}}
\newcommand{\E}{\mathbb{E}}
\newcommand{\1}{\textbf{1}}
\newcommand{\R}{\mathbb{R}}
\newcommand{\C}{\mathbb{C}} 
\newcommand{\Z}{\mathbb{Z}}
\newcommand{\N}{\mathbb{N}}
\newcommand{\p}[1]{\mathbb{P}\left( #1 \right)}
\newcommand{\scal}[2]{\left\langle #1, #2 \right\rangle}
\newcommand{\red}{\color{red}}
\newcommand{\shift}{\vdash}

\DeclareMathOperator{\Var}{Var}
\DeclareMathOperator{\sgn}{sgn}

\usepackage[paper=a4paper, left=1.3in, right=1.3in, top=1in, bottom=1in]{geometry}
\linespread{1.3}
\pagestyle{plain}

\newtheorem{theorem}{Theorem}
\newtheorem{lemma}[theorem]{Lemma}
\newtheorem{corollary}[theorem]{Corollary}

\theoremstyle{remark}
\newtheorem{remark}[theorem]{Remark}


\newtheorem{conjecture}{Conjecture}

\theoremstyle{definition}
\newtheorem{defn}[theorem]{Definition}
\newtheorem{exmp}[theorem]{Example}


\title{\vspace{-3em}Log}



\begin{document}

\section{8/20}

We use $\vdash$ to mean equality up to a shift(multiplication by $z^{\alpha}$).

\subsection{Quadratic Forms}

We consider whether $\sum_{i,j} a_{ij} \epsilon_i \epsilon_j = \langle A \epsilon,\epsilon\rangle$ is type L for vector of independently distributed random signs $\epsilon$ and matrix $A$. 

Compute $\sum_{i,j} a_{ij} \epsilon_i \epsilon_j = \sum_{1 \leq i,j\leq n} a_{ij} \epsilon_i \epsilon_j + \sum_{i=1}^n \epsilon_{n+1}\epsilon_i(a_{ij}+a_{ji}) = \sum_{1 \leq i,j\leq n} a_{ij} \epsilon_i \epsilon_j+ \sum_{i=1}^n \epsilon_{n+1}\epsilon_i(b_i)$. 

Then $\E z^{\langle A\epsilon, \epsilon \rangle} = \E z^{\sum_{1 \leq i,j\leq n} a_{ij} \epsilon_i \epsilon_j+ \sum_{i=1}^n \epsilon_{n+1}\epsilon_i(b_i)} =$ 

$\frac{1}{2}\E_{\epsilon_{[n]}}z^{\sum_{1 \leq i,j\leq n} a_{ij} \epsilon_i \epsilon_j+ \sum_{i=1}^n \epsilon_i(b_i)}+\frac{1}{2}\E_{\epsilon_{[n]}}z^{\sum_{1 \leq i,j\leq n} a_{ij} \epsilon_i \epsilon_j- \sum_{i=1}^n \epsilon_i(b_i)}=$

$\E z^{\sum_{1 \leq i,j\leq n} a_{ij} \epsilon_i \epsilon_j+ \sum_{i=1}^n \epsilon_i(b_i)}$ via symmetry. Since we are only interested in the zeroes we may factor out constant terms $a_{i,i}\epsilon_i$ to examine $\E z^{\sum_{1 \leq i<j\leq n} c_{ij} \epsilon_i \epsilon_j+ \sum_{i=1}^n \epsilon_i(b_i)}$ where $c_{ij} = a_{ij}+a_{ji}$. Wlog we may suppose all coefficients $> 0$.

Consider the case $n = 2$. We have $\E z^{a\epsilon_1 \epsilon_2 + b\epsilon_1 + c \epsilon_2} \propto z^{a+b+c} + z^{-a+b-c}+z^{-a-b+c}+z^{a-b-c}$. After a shift this becomes $z^{2a+2b+2c}+z^{2a}+z^{2b}+z^{2c}$. In the integer case, the polynomial must be palindromic. If $a=b=c$ then we have $z^{6a} +3z^{2a} \vdash z^{4a} +3$ which is not palindromic. Suppose wlog $a=b$. Then $z^{2a+2b+2c}+z^{2a}+z^{2b}+z^{2c} = z^{4a+2c}+2z^{2a}+z^{2c}$. Then to be palindromic it must be that $2a = \frac{1}{2}(4a+2c)$ implying $c = 0$. Note if this is the case the polynomial is unirooted. Now suppose wlog  $a<b<c$. We have $z^{2a+2b+2c}+z^{2a}+z^{2b}+z^{2c} \shift z^{2b+2c}+z^{2(c-a)}+z^{2(b-a)}+1$. For palindromicity we need $2b+2c = 2c+2b-4a \implies a = 0$. So again we require the smallsest term to be 0. So $z^{2(b+c)}+z^{2b}+z^{2c}+1$. Note this does have all roots on unit circle then.

In summary: All three terms cannot be equal. If two are equal the other must be zero. If none are equal the smallest must be 0. Note that if any of the terms are 0, the result is trivially type l(as it is either a sum of two type L random variables or a symmerization of one).

Further note in general the diagonal of $A$, $(a_{ii})_{i=1}^n$ is irrelevant due to shifting. So in particular Positive semi definiteness is not required. Since $tr(A) = \sum \lambda_i$ where the trace is arbitrary.

\section{9/13}

\subsection{Polya Schur Theory}



\begin{thebibliography}{9}


\bibitem{MR} Markovsky. I, Shodhan. R,
Palindromic Polynomials, Time-Reversible Systems, and Conserved Quantities. https://eprints.soton.ac.uk/266592/1/Med08.pdf

\bibitem{KB} Keel. L, Bhattarcharyya. S,
A New Proof of the Jury Test. https://ieeexplore.ieee.org/document/703305
%\bibitem{Z-2} 
%https://mathoverflow.net/questions/208349

\end{thebibliography}

\end{document}



